
\documentclass{IOS-Book-Article}

\usepackage{mathptmx}
\usepackage{soul}\setuldepth{article}
\usepackage{graphicx}
\graphicspath{ {./figures/} }
%\usepackage{times}
%\normalfont
%\usepackage[T1]{fontenc}
%\usepackage[mtplusscr,mtbold]{mathtime}
%
\def\hb{\hbox to 11.5 cm{}}

\begin{document}

\pagestyle{headings}
\def\thepage{}
\begin{frontmatter}              % The preamble begins here.


%\pretitle{Pretitle}
\title{First Steps towards a Risk of Bias Corpus of Randomized Controlled Trials}

\markboth{}{April 2022\hb}
%\subtitle{Subtitle}

\author[A,B]{\fnms{Anjani} \snm{Dhrangadhariya}\orcid{0000-0003-1691-1338}%
\thanks{Corresponding Author: Anjani Dhrangadhariya, Informatics Institute, HES-SO Valais-Wallis, Technopole 3,
3960 Sierre, Switzerland; E-mail:
anjani.dhrangadhariya@hevs.ch.}},
\author[C,D]{\fnms{Roger} \snm{Hilfiker}}
,
\author[C,D]{\fnms{Martin} \snm{Sattlemayer}}
,
\author[C,D]{\fnms{Katia} \snm{Giacomino}}
,
\author[C,D]{\fnms{Rahel} \snm{Caliesch}}
,
\author[C,D]{\fnms{Simone} \snm{Elsig}}
,
\author[E,F]{\fnms{Nona} \snm{Naderi}}
and
\author[A,B]{\fnms{Henning} \snm{M\"uller}}

\runningauthor{A. Dhrangadhariya et al.}
\address[A]{Informatics Institute, HES-SO Valais-Wallis, Sierre, Switzerland}
\address[B]{University of Geneva (UNIGE), Geneva, Switzerland}
\address[C]{School of Health Sciences, HES-SO Valais-Wallis, Leukerbad, Switzerland.}
\address[D]{Department of Physiotherapy, HES-SO Valais-Wallis, Leukerbad, Switzerland.}
\address[E]{Geneva School of Business Administration, HES-SO Geneva, Switzerland.}
\address[F]{SIB Swiss Institute of Bioinformatics (SIB), Geneva, Switzerland}

\begin{abstract}
Risk of bias (RoB) assessment of randomized clinical trials (RCTs) is vital to conducting systematic reviews. 
Manual RoB assessment for hundreds of RCTs is a cognitively demanding, lengthy process and is prone to subjective judgment. 
Supervised machine learning (ML) can help to accelerate this process but requires a hand-labelled corpus.
There are currently no RoB annotation guidelines for randomized clinical trials or any annotated corpora.
In this pilot project, we test the practicality of directly using the revised Cochrane RoB 2.0 tool for developing an RoB annotated corpus with a novel multi-level annotation scheme.
We report inter-annotator agreement among four experienced annotators, which ranges between 0\% for some bias classes and 76\% for others.
Finally, we discuss the shortcomings of this direct translation of annotation guidelines and scheme and suggest approaches to improve them to obtain an RoB annotated corpus suitable for ML.
\end{abstract}

\begin{keyword}
risk of bias\sep annotation\sep
systematic reviews\sep corpus\sep automation
\end{keyword}
\end{frontmatter}
\markboth{April 2022\hb}{April 2022\hb}
%\thispagestyle{empty}
%\pagestyle{empty}
%
%
%
\section{Introduction}
\label{sec:intro}
%
Systematic reviews (SRs) synthesized from randomized controlled trials (RCTs) are the highest quality of evidence in the evidence hierarchy.
In theory, an RCT accurately measures the treatment effect on patient outcomes but can be biased in practice due to flawed study design, execution, analysis or outcome reporting.~\cite{hariton2018randomised}
Biases in RCTs cannot be measured, but the risk of bias (RoB) can be assessed.
The researchers conducting SRs must rigorously look for possible biases before incorporating them into SRs.
Published RCTs are exponentially increasing~\footnote{\url{https://pubmed.ncbi.nlm.nih.gov/?term=randomized\%20controlled\%20trial&filter=pubt.randomizedcontrolledtrial}} over time, so manual assessment for every study becomes a protracted process.
Machine learning (ML) can help accelerate this process by directly pointing the reviewers to the parts of the text relevant to identifying RoB, leading to quickly judging the trial quality.
Both Marshall \textit{et al.} and Millard \textit{et al.} attempted automated RoB assessment, albeit using proprietary, pay-walled data.~\cite{marshall2015automating,millard2016machine}
Recently, Wang \textit{et al.} released a hand-labelled RoB corpus for animal studies using a preclinical RoB assessment tool.~\cite{wang2022risk}
RoB assessment is a knowledge-heavy task where even highly trained experts are prone to subjective judgments.
The primary requirement to develop such a corpus entails creating a clear-cut annotation scheme and guidelines.
As neither exists, this work focuses on two primary concerns.
1) To test whether the widely used revised Cochrane's RoB 2.0 tool for RCTs (RoB 2.0) could be used as RoB annotation guidelines to develop a corpus that could be used for training supervised ML models.~\cite{lansbury2020co}
2) To develop and test an RoB annotation scheme that closely mimics the RoB 2.0.~\cite{sterne2019rob}
%
%
%
\section{Methods}
\label{sec:methods}
%
\subsection{Formulating Annotation Scheme}
%
%We formulate the annotation scheme as a function of the RoB 2.0 tool.
RoB 2.0 tool divides biases into five risk domains which further decompose into several signalling questions, each corresponding to different parts of the trial design.
Each signalling question prompts the reviewer to look for a piece(s) of factual evidence in the RCT to respond with one of the five response options: ``Yes'', ``Probably yes'', ``No'', ``Probably no'', or ``No information''.
E.g. to respond to the signalling question ``Was the allocation sequence random?'', the reviewers need to identify whether a proper methodology was used for random participant allocation and only if a proper methodology is identified the reviewer responds to this question as ``Yes'' and otherwise ``No''.

We formulated an annotation scheme where each signalling question is an entity.
Each entity has five entity labels corresponding to the five response options to that question.
Entities represent the factual evidence from the RCTs, and the entity labels incorporate the reviewer's risk judgment.
The cumulative risk judgment (``Low-risk'', ``High-risk'' or ``Some-concerns'') from each risk domain was estimated using decision flowcharts in RoB 2.0, combining the responses from all the signalling questions.
To account for this, we had an additional document-level annotation for each risk domain whereby a reviewer chose either of the three risk judgments.
%
%
%
\subsection{Preliminary Annotation guidelines}
\label{subsec:annot_guide}
%
Full-text RCTs were annotated using the RoB assessment instructions from the RoB 2.0.~\footnote{\url{https://drive.google.com/file/d/19R9savfPdCHC8XLz2iiMvL_71lPJERWK/view}}
A.D. developed the generic annotation guidelines with four physiotherapists experienced in bias assessment to ensure consistency.
Complete sentence(s) or phrase(s) were annotated depending upon the text parts relevant to answering a signalling question.
All the text information pertinent to answering a question was marked, even if the information was found in different parts of the full text.
Table or figure captions relevant to answering were marked.
If the information was not found in the captions, it was marked within the table contents.
If a reference to the table or figure led to answering the question, it was annotated.
If all three were relevant then all were annotated.
%
\subsection{Pilot annotation}
\label{subsec:annotation}
%
The four physiotherapists (R.H., M.S., K.G., R.C.) did the pilot annotation on ten RCTs using the RoB 2.0, generic annotation guidelines and the developed annotation scheme.
These 10 RCTs were selected in the following way.
An Entrez~\footnote{The Entrez Global Query Cross-Database Search System is a federated search engine or web portal that allows users to search PubMed database} search using the search query ``{\fontfamily{qcr}\selectfont (randomized[title] or randomized[title]) and (rehabilitation or (physical therapy))}'' was performed ten times to retrieve studies from one-year time-spans each between 2000 - 2019.
Each query was restricted to retrieve 1000 documents, of which ten were randomly chosen for each time span.
Of the ten sampled studies, R.H. took the first possible study with a freely available PDF.
R.H. and M.S. are professors and associate professors, respectively, with bias assessment experience in physiotherapy and rehabilitation.
K.G. and R.C. are doctoral researchers with experience conducting RoB ratings in several S.R.s.


Tagtog, a commercial text annotation tool, was used for annotating the full-text PDFs (Portable Document Format)~\cite{cejuela2014tagtog}.
Annotators were given a brief training session with Tagtog.
The task for the annotator was to annotate text relevant to answering each signalling question entity, choose a judgment response option, and finally select the risk domain judgment for the five risk domains.

We report token-level pairwise F1 measure as an inter-annotator agreement (IAA) for the entity $IAA_{sq}$ and entity label $IAA_{response}$
annotations.~\cite{brandsen2020creating,deleger2012building}
$IAA_{sq}$ and $IAA_{response}$ measure how reliable the RoB 2.0 guidelines were for selecting the same parts of the text to answer signalling questions.
Cohen's Kappa was chosen for the document-level $IAA_{rd}$ agreements.~\cite{mchugh2012interrater}
%
%
%
\section{Results}
\label{sec:results}
%
%Therefore, each entity could have one of the five response options incorporating the reviewer's judgment of the answer.
%The reviewer needs to mark the identified text evidence (a phrase, sentence (s), or paragraph) with the RoB entity along with one of the five response options.
%We have a hierarchical annotation scheme comprising 22 entities corresponding to the 22 signalling questions, each with five response options.


A visual representation of our annotation scheme is illustrated in Figure~\ref{fig:ann_scheme}. 
%
\begin{figure}[!htbp]
    \centering
    \includegraphics[width=0.90\textwidth]{Figures/annotation_scheme.pdf}
    \caption{Annotation scheme. I. signalling question annotation scheme: each signalling question (RoB 1.1, 1.2, 1.3, 2.1, ...) is an entity/class that could take either of five response options or attributes. II. Risk domain annotation scheme: each risk domain (RoB 1, 2, 3, 4, and 5) is a class that could take either of three risk classes based on the assessment of RoB signalling questions. This final risk domain judgment depends upon instructions from RoB 2.0 guidelines.}
    \label{fig:ann_scheme}
\end{figure}
%

\begin{table}[!ht]
    \centering
    \begin{tabular}{|l||l|l|l|l|l|l|l||l|l|l|l|l|}
    \hline
        SQ & P1 & P2 & P3 & P4 & P5 & P6 & Avg & Y & PY & NI & PN & N \\ \hline \hline
        1.1 & 23.1 & 24.5 & 52.2 & 57.0 & 48.0 & 21.5 & 37.7 & 21.8 & 7.1 & 0.0 & - & - \\ 
        1.2 & 66.1 & 50.3 & 72.8 & 50.7 & 46.0 & 50.5 & 56.1 & 4.9 & 11.5 & 10.2 & 0.0 & - \\ 
        1.3 & 69.5 & 20.5 & 16.1 & 31.6 & 59.9 & 53.5 & 41.8 & - & - & 41.8 & 11.4 & 9.9 \\ \hline
        2.1 & 1.0 & 1.4 & 0.0 & 9.1 & 19.1 & 0.0 & 5.1 & 8.2 & 0.0 & - & 3.0 & 0.0 \\ 
        2.2 & 18.3 & 7.3 & 11.1 & 0.0 & 23.0 & 7.4 & 11.2 & 3.6 & 0.0 & 0.0 & 0.0 & 0.0 \\ 
        2.3 & 20.6 & 5.5 & 13.4 & 0.0 & 0.0 & 0.0 & 6.6 & - & 0.0 & - & 1.0 & 0.0 \\ 
        2.4 & 0 & - & - & 0 & 0 & - & 0 & - & 0 & - & 0 & - \\ 
        2.5 & 0 & 0 & 0 & 0 & 0 & - & 0 & 0 & 0 & - & 0 & - \\ 
        2.6 & 75.3 & 68.9 & 19.3 & 63.9 & 12.9 & 19.6 & 43.3 & 39.4 & 0.0 & 0.0 & 0.0 & 3.6 \\ 
        2.7 & 0.0 & 6.6 & 0.0 & 0.0 & 0.0 & 0.0 & 1.1 & 0.0 & 0.0 & - & 0.0 & 0.0 \\ \hline
        3.1 & 45.8 & 23.6 & 32.2 & 43.4 & 22.9 & 14.8 & 30.4 & 47.6 & 0.6 & - & 1.3 & 3.3 \\ 
        3.2 & 1.4 & 0.0 & 0.0 & 3.3 & 7.4 & 0.9 & 2.2 & 0.0 & 0.0 & - & 0.0 & 0.0 \\ 
        3.3 & 0.0 & 0.0 & 0.0 & 16.4 & 0.0 & 0.0 & 2.8 & - & 0.0 & 31.4 & 0.0 & 0.0 \\ 
        3.4 & - & 0 & - & 0 & 0 & 0 & 0 & 0 & 0 & 0 & 0 & 0 \\ \hline
        4.1 & 4.0 & 6.6 & 14.2 & 25.6 & 22.3 & 6.3 & 13.2 & - & - & - & 0.8 & 12.0 \\ 
        4.2 & 1.8 & 0.0 & 0.4 & 0.0 & 40.1 & 0.0 & 7.1 & - & - & - & 0.3 & 0.0 \\
        4.3 & 7.6 & 13.9 & 5.0 & 10.5 & 39.5 & 8.4 & 14.2 & 0.0 & 0.0 & 0.0 & 13.1 & 20.5 \\ 
        4.4 & 0 & 0 & 0 & 0 & 0 & 0 & 0 & 0 & 0 & - & 0 & 0 \\ 
        4.5 & 0 & 0 & 0 & 0 & 0 & 0 & 0 & 0 & 0 & - & 0 & - \\ \hline
        5.1 & 0.0 & 0.0 & 0.0 & 0.0 & 0.0 & 4.2 & 0.7 & 0.0 & 0.0 & 0.0 & 0.0 & 0.0 \\ 
        5.2 & 23.9 & 0.0 & 0.0 & 0.0 & 0.0 & 2.4 & 4.4 & - & 0.0 & 0.0 & 0.0 & 0.0 \\ 
        5.3 & 0.2 & 0.0 & 0.0 & 0.4 & 8.1 & 42.0 & 8.4 & - & 0.0 & 0.6 & 0.0 & 0.0 \\ \hline
    \end{tabular}
    
\end{table}

%
%
%
\section{Discussion}

\section{Conclusion}
\label{sec:conclusion}
%
In conclusion, revised Cochrane's RoB assessment 2.0 tool cannot be directly used as RoB corpus annotation guidelines.
The annotation scheme directly adapted from RoB 2.0 document also needs improvement, as detailed in the discussion section.
The annotation scheme directly adapted from RoB 2.0 document also needs improvement, as detailed in the discussion section.
We use insights from this pilot project to develop crisp annotation guidelines to obtain consistent annotations.
The annotated RCTs from this study are available on Zenodo.
%
%
%
%%%%% References %%%%%
\bibliographystyle{spiebib} 
%
%
\bibliography{bibliography}
%
\end{document}
